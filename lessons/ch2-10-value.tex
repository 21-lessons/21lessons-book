\chapter{Valor}
\label{les:10}

\begin{chapquote}{Lewis Carroll, \textit{Alice no País das Maravilhas}}
\enquote{Era o Coelho Branco, voltando vagarosamente, olhando ansiosamente para trás, como se tivesse perdido algo\ldots}
\end{chapquote}

O valor é um tanto paradoxal, e existem várias teorias \footnote{Ver \textit {Teoria do valor (economia)} na Wikipedia \cite{wiki:theory-of-value}} que tentam explicar por que valorizamos certas coisas ao invés de outras. As pessoas estão cientes desse paradoxo há milhares de anos. Como Platão escreveu em seu diálogo com Eutidemo, valorizamos algumas coisas porque são raras, e não apenas por sua necessidade para nossa sobrevivência.

\begin{quotation}\begin{samepage}
\enquote{E se você for prudente, dará este mesmo conselho a seus alunos também --- que eles nunca devem conversar com ninguém, exceto você e uns aos outros. Pois é o raro, Eutidemo, que é precioso, enquanto a água é mais barata, embora seja a melhor, como disse Píndaro.}
\begin{flushright} -- Platão\footnote{Platão, \textit{Eutidemo} \cite{euthydemus}}
\end{flushright}\end{samepage}\end{quotation}

Este paradoxo do valor \footnote{Veja \textit {Paradoxo do valor} na Wikipedia \cite {wiki:paradox-of-value}} mostra algo interessante sobre nós, seres humanos: parecemos valorizar as coisas de uma forma subjetiva \footnote{Veja \textit{Teoria subjetiva do valor} na Wikipedia \cite{wiki:subjective-theory-of-value}}, mas fazemos isso com certos critérios não arbitrários. Algo pode ser \textit{precioso} para alguém por uma variedade de razões, mas as coisas que todos valorizamos compartilham certas características. Se pudermos copiar algo com muita facilidade, ou se for naturalmente abundante, não a valorizamos.

Parece que valorizamos algo porque é escasso (ouro, diamantes, tempo), difícil ou trabalhoso de ser produzido, por não poder ser substituído (uma velha fotografia de um ente querido), por ser útil de uma forma que permite que façamos coisas que de outra forma não poderíamos, ou uma combinação entre essas características, como grandes obras de arte.

O Bitcoin é tudo isso: é extremamente raro (21 milhões), cada vez mais difícil de produzir (redução da recompensa através dos halvings), não pode ser substituído (uma chave privada perdida é perdida para sempre) e nos permite fazer algumas coisas bastante úteis com ele. É indiscutivelmente a melhor ferramenta para transferência de valor entre fronteiras, virtualmente resistente à censura e confisco no processo, além de ser uma reserva de valor auto-soberana, permitindo que os indivíduos armazenem sua riqueza independentemente de bancos e governos, apenas citando dois interessados em proibir isso.

\paragraph{O Bitcoin me ensinou que o valor é subjetivo, mas não arbitrário.}

% ---
%
% #### Down the Rabbit Hole
%
% - [Euthydemus] by Plato
% - [Theory of Value][multiple theories], [Paradox of Value][paradox of value], [Subjective Theory of Value][subjective] on Wikipedia
%
% [Euthydemus]: http://www.perseus.tufts.edu/hopper/text?doc=Perseus:text:1999.01.0178:text=Euthyd.
% [Plato]: http://www.perseus.tufts.edu/hopper/text?doc=plat.+euthyd.+304b
%
% <!-- Wikipedia -->
% [multiple theories]: https://en.wikipedia.org/wiki/Theory_of_value_%28economics%29
% [paradox of value]: https://en.wikipedia.org/wiki/Paradox_of_value
% [subjective]: https://en.wikipedia.org/wiki/Subjective_theory_of_value
% [alice]: https://en.wikipedia.org/wiki/Alice%27s_Adventures_in_Wonderland
% [carroll]: https://en.wikipedia.org/wiki/Lewis_Carroll
