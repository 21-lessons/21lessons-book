---
layout: lesson
title: Lesson 10
subtitle: Value
quote: "It was the white rabbit, trotting slowly back again, and looking anxiously about it as it went, as if it had lost something..."
categories: [bitcoin, lesson]
audio: /assets/audio/21lessons/2-10.m4a 
---

Value is somewhat paradoxical, and there are [multiple theories] which
try to explain why we value certain things over other things. People
have been aware of this paradox for thousands of years. As Plato wrote
in his dialogue with Euthydemus, we value some things because they are
rare, and not merely based on their necessity for our survival.

> "And if you are prudent you will give this same counsel to your pupils
> also --- that they are never to converse with anybody except you and
> each other. For it is the rare, Euthydemus, that is precious, while
> water is cheapest, though best, as Pindar said."
> <cite>[Plato]</cite>

This [paradox of value] shows something interesting about us humans: we
seem to value things on a [subjective] basis, but do so with certain
non-arbitrary criteria. Something might be *precious* to us for a
variety of reasons, but things we value do share certain
characteristics. If we can copy something very easily, or if it is
naturally abundant, we do not value it.

It seems that we value something because it is scarce (gold, diamonds,
time), difficult or labor-intensive to produce, can't be replaced (an
old photograph of a loved one), is useful in a way in which it enables
us to do things which we otherwise couldn't, or a combination of those,
such as great works of art.

Bitcoin is all of the above: it is extremely rare (21 million),
increasingly hard to produce (reward halvening), can't be replaced (a
lost private key is lost forever), and enables us to do some quite
useful things. It is arguably the best tool for value transfer across
borders, virtually resistant to censorship and confiscation in the
process, plus, it is a self-sovereign store of value, allowing
individuals to store their wealth independent of banks and governments,
just to name two.

Bitcoin taught me that value is subjective but not arbitrary.

---

#### Down the Rabbit Hole

- [Euthydemus] by Plato
- [Theory of Value][multiple theories], [Paradox of Value][paradox of value], [Subjective Theory of Value][subjective] on Wikipedia

[Euthydemus]: http://www.perseus.tufts.edu/hopper/text?doc=Perseus:text:1999.01.0178:text=Euthyd.
[Plato]: http://www.perseus.tufts.edu/hopper/text?doc=plat.+euthyd.+304b

<!-- Wikipedia -->
[multiple theories]: https://en.wikipedia.org/wiki/Theory_of_value_%28economics%29
[paradox of value]: https://en.wikipedia.org/wiki/Paradox_of_value
[subjective]: https://en.wikipedia.org/wiki/Subjective_theory_of_value
[alice]: https://en.wikipedia.org/wiki/Alice%27s_Adventures_in_Wonderland
[carroll]: https://en.wikipedia.org/wiki/Lewis_Carroll
