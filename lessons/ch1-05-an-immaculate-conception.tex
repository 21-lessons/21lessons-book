\chapter{Uma concepção imaculada}
\label{les:5}

\begin{chapquote}{Lewis Carroll, \textit{Alice no País das Maravilhas}}
\enquote{Suas cabeças se foram, para servi-la, Majestade}, os soldados gritaram em resposta\ldots
\end{chapquote}

Todo mundo adora uma boa história de origem. A história de origem do Bitcoin é fascinante, e os detalhes dela são mais importantes do que se possa pensar quando sabemos pela primeira vez. Quem é Satoshi Nakamoto? Ele era uma pessoa ou um grupo de pessoas? Ele era ela? Seria um alien viajando no tempo ou IA avançada? Deixando de lado as teorias estranhas, provavelmente nunca saberemos. E isso é importante.

O Satoshi escolheu ser anônimo. Ele plantou a semente do Bitcoin. Ele ficou por aqui por tempo suficiente para garantir que a rede não morresse quando estava engatinhando. E então ele desapareceu.

O que pode parecer um golpe estranho de anonimato é realmente crucial para um sistema verdadeiramente descentralizado. Sem controle centralizado. Nenhuma autoridade centralizada. Nenhum inventor. Ninguém para processar, torturar, chantagear ou extorquir. Uma concepção imaculada de tecnologia.

\begin{quotation}\begin{samepage}
\enquote{Uma das melhores coisas que Satoshi fez foi desaparecer.}
\begin{flushright} -- Jimmy Song\footnote{Jimmy Song, \textit{Por que o Bitcoin é Diferente} \cite{bitcoin-different}}
\end{flushright}\end{samepage}\end{quotation}

\newpage

Desde o nascimento do Bitcoin, milhares de outras criptomoedas foram criadas. Nenhum desses clones compartilha sua história de origem. Se você quiser substituir o Bitcoin, terá que transcender sua história de origem. Em uma guerra de ideias, as narrativas ditam a sobrevivência.

\begin{quotation}\begin{samepage}
\enquote{O ouro foi transformado pela primeira vez em joias e usado para troca há mais de 7.000 anos. O brilho cativante do ouro o levou a ser considerado um presente
dos deuses.}
\begin{flushright} Austrian Mint\footnote{The Austrian Mint, \textit{Gold: The Extraordinary Metal} \cite{gold-gift-gods}}
\end{flushright}\end{samepage}\end{quotation}

Como o ouro nos tempos antigos, o Bitcoin pode ser considerado um presente dos deuses. Ao contrário do ouro, as origens dos Bitcoins são muito humanas. E desta vez, sabemos quem são os deuses do desenvolvimento e da manutenção: pessoas de todo o mundo, anônimas ou não.

\paragraph{O Bitcoin me ensinou que narrativas são importantes.}

% ---
%
% #### Down the Rabbit Hole
%
% - [Why Bitcoin is different][Jimmy Song] by Jimmy Song
% - [Gold: The Extraordinary Metal] by the Austrian Mint
%
% <!-- Down the Rabbit Hole -->
% [Jimmy Song]: https://medium.com/@jimmysong/why-bitcoin-is-different-e17b813fd947
% [Gold: The Extraordinary Metal]: https://www.muenzeoesterreich.at/eng/discover/for-investors/gold-the-extraordinary-metal
%
% <!-- Wikipedia -->
% [alice]: https://en.wikipedia.org/wiki/Alice%27s_Adventures_in_Wonderland
% [carroll]: https://en.wikipedia.org/wiki/Lewis_Carroll
