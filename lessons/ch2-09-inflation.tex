\chapter{Inflation}
\label{les:9}

\begin{chapquote}{The Queen of Hearts} % TODO: check if it's the queen or not
\enquote{My dear, here we must run as fast as we can, just to stay in place. And if you
wish to go anywhere you must run twice as fast as that.}
\end{chapquote}

Trying to understand monetary inflation, and how a non-inflationary
system like Bitcoin might change how we do things, was the starting
point of my venture into economics. I knew that inflation was the rate
at which new money was created, but I didn't know too much beyond that.

While some economists argue that inflation is a good thing, others argue
that \enquote{hard} money which can't be inflated easily --- as we had in the
days of the gold standard --- is essential for a healthy economy.
Bitcoin, having a fixed supply of 21 million, agrees with the latter
camp.

Usually, the effects of inflation are not immediately obvious. Depending
on the inflation rate (as well as other factors) the time between cause
and effect can be several years. Not only that, but inflation affects
different groups of people more than others. As Henry Hazlitt points out
in \textit{Economics in One Lesson}: \enquote{The art of economics consists in looking
not merely at the immediate but at the longer effects of any act or
policy; it consists in tracing the consequences of that policy not
merely for one group but for all groups.}

One of my personal lightbulb moments was the realization that issuing
new currency --- printing more money --- is a \textit{completely} different
economic activity than all the other economic activities. While real
goods and real services produce real value for real people, printing
money effectively does the opposite: it takes away value from everyone
who holds the currency which is being inflated.

\begin{quotation}\begin{samepage}
\enquote{Mere inflation --- that is, the mere issuance of more money, with the
consequence of higher wages and prices --- may look like the creation
of more demand. But in terms of the actual production and exchange of
real things it is not.}
\begin{flushright} -- Henry Hazlitt\footnote{Henry Hazlitt, \textit{Economics in One Lesson} \cite{hazlitt}}
\end{flushright}\end{samepage}\end{quotation}

The destructive force of inflation becomes obvious as soon as a little inflation
turns into \textit{a lot}. If money hyperinflates things get ugly real
quick.\footnote{\url{https://en.wikipedia.org/wiki/Hyperinflation}
\cite{wiki:hyperinflation}} As the inflating currency falls apart, it will fail
to store value over time and people will rush to get their hands on any goods
which might do.

\paragraph{}
Another consequence of hyperinflation is that all the money which people
have saved over the course of their life will effectively vanish. The
paper money in your wallet will still be there, of course. But it will
be exactly that: worthless paper.

\begin{figure}
  \includegraphics[width=\textwidth]{assets/images/children-playing-with-money.png}
  \caption{Hyperinflation in the Weimar Republic (1921-1923)}
  \label{fig:children-playing-with-money}
\end{figure}

\paragraph{}
Money declines in value with so-called \enquote{mild} inflation as well. It
just happens slowly enough that most people don't notice the diminishing
of their purchasing power. And once the printing presses are running,
currency can be easily inflated, and what used to be mild inflation
might turn into a strong cup of inflation by the push of a button. As
Friedrich Hayek pointed out in one of his essays, mild inflation usually
leads to outright inflation.

\begin{quotation}\begin{samepage}
\enquote{`Mild' steady inflation cannot help --- it can lead only to outright
inflation.}
\begin{flushright} -- Friedrich Hayek\footnote{Friedrich Hayek, \textit{1980s
Unemployment and the Unions} \cite{hayek-inflation}}
\end{flushright}\end{samepage}\end{quotation}

Inflation is particularly devious since it favors those who are closer
to the printing presses. It takes time for the newly created money to
circulate and prices to adjust, so if you are able to get your hands on
more money before everyone else's devaluates you are ahead of the
inflationary curve. This is also why inflation can be seen as a hidden
tax because in the end governments profit from it while everyone else
ends up paying the price.

\begin{quotation}\begin{samepage}
\enquote{I do not think it is an exaggeration to say history is largely a
history of inflation, and usually of inflations engineered by
governments for the gain of governments.}
\begin{flushright} -- Friedrich Hayek\footnote{Friedrich Hayek, \textit{Good Money} \cite{hayek-good-money}}
\end{flushright}\end{samepage}\end{quotation}

\newpage

So far, all government-controlled currencies have eventually been
replaced or have collapsed completely. No matter how small the rate of
inflation, \enquote{steady} growth is just another way of saying exponential
growth. In nature as in economics, all systems which grow exponentially
will eventually have to level off or suffer from catastrophic collapse.

\paragraph{}
\enquote{It can't happen in my country,} is what you're probably thinking. You don't
think that if you are from Venezuela, which is currently suffering from
hyperinflation. With an inflation rate of over 1 million percent, money is
basically worthless. \cite{wiki:venezuela}

\paragraph{}
It might not happen in the next couple of years, or to the particular currency
used in your country. But a glance at the list of historical
currencies\footnote{See \textit{List of historical currencies} on Wikipedia.
\cite{wiki:historical-currencies}} shows that it will inevitably happen over a
long enough period of time. I remember and used plenty of those listed: the
Austrian schilling, the German mark, the Italian lira, the French franc, the
Irish pound, the Croatian dinar, etc. My grandma even used the Austro-Hungarian
Krone. As time moves on, the currencies currently in use\footnote{See
\textit{List of currencies} on Wikipedia \cite{wiki:list-of-currencies}} will
slowly but surely move to their respective graveyards. They will hyperinflate or
be replaced. They will soon be historical currencies. We will make them
obsolete.

\begin{quotation}\begin{samepage}
\enquote{History has shown that governments will inevitably succumb to the
temptation of inflating the money supply.}
\begin{flushright} -- Saifedean Ammous\footnote{Saifedean Ammous, \textit{The Bitcoin
Standard} \cite{bitcoin-standard}}
\end{flushright}\end{samepage}\end{quotation}

\newpage

Why is Bitcoin different? In contrast to currencies mandated by the government,
monetary goods which are not regulated by governments, but by the laws of
physics\footnote{Gigi, \textit{Bitcoin's Energy Consumption - A shift in
perspective} \cite{gigi:energy}}, tend to survive and even hold their respective
value over time. The best example of this so far is gold, which, as the
aptly-named \textit{Gold-to-Decent-Suit Ratio}\footnote{History shows that the
price of an ounce of gold equals the price of a decent men's suit, according to Sionna
investment managers \cite{web:gold-to-decent-suite-ratio}} shows, is holding its
value over hundreds and even thousands of years. It might not be perfectly
\enquote{stable} --- a questionable concept in the first place --- but the value it
holds will at least be in the same order of magnitude.

If a monetary good or currency holds its value well over time and space,
it is considered to be \textit{hard}. If it can't hold its value, because it
easily deteriorates or inflates, it is considered a \textit{soft} currency. The
concept of hardness is essential to understand Bitcoin and is worthy of
a more thorough examination. We will return to it in the last economic
lesson: sound money.

\paragraph{}
As more and more countries suffer from
hyperinflation more and more people will have to face the reality
of hard and soft money. If we are lucky, maybe even some central bankers will be
forced to re-evaluate their monetary policies. Whatever might happen, the
insights I have gained thanks to Bitcoin will probably be invaluable, no matter
the outcome.

\paragraph{Bitcoin taught me about the hidden tax of inflation and the catastrophe
of hyperinflation.}

% ---
%
% #### Down the Rabbit Hole
%
% - [Economics in One Lesson][Henry Hazlitt] by Henry Hazlitt
% - [1980's Unemployment and the Unions][unions] by Friedrich Hayek
% - [Good Money, Part II][good-money]: Volume Six of the Collected Works of F.A. Hayek
% - [The Bitcoin Standard] by Saifedean Ammous
% - [Hyperinflation][hyperinflates], [economic crisis in Venezuela][wiki-venezuela], [list of historical currencies], [list of currencies][currently in use] on Wikipedia
%
% [unions]: https://books.google.com/books/about/1980s_unemployment_and_the_unions.html?id=xM9CAQAAIAAJ
% [good-money]: https://books.google.com/books?id=l_A1vVIaYBYC
%
% [Henry Hazlitt]: https://mises.org/library/economics-one-lesson
% [hyperinflates]: https://en.wikipedia.org/wiki/Hyperinflation
% [inflation cannot help]: https://books.google.com/books?id=zZu3AAAAIAAJ&dq=%22only+while+it+accelerates%22&focus=searchwithinvolume&q=%22steady+inflation+cannot+help%22
% [history of inflation]: https://books.google.com/books?id=l_A1vVIaYBYC&pg=PA142&dq=%22history+is+largely+a+history+of+inflation%22&hl=en&sa=X&ved=0ahUKEwi90NDLrdnfAhUprVkKHUx1CmIQ6AEIKjAA#v=onepage&q=%22history%20is%20largely%20a%20history%20of%20inflation%22&f=false
% [wiki-venezuela]: https://en.wikipedia.org/wiki/Crisis_in_Venezuela#Economic_crisis
% [by the laws of physics]: https://link.medium.com/9fzq2L0J3S
% [\textit{Gold-to-Decent-Suit Ratio}]: https://www.businesswire.com/news/home/20110819005774/en/History-Shows-Price-Ounce-Gold-Equals-Price
% [The Bitcoin Standard]: https://thesaifhouse.wordpress.com/book/
%
% <!-- Wikipedia -->
% [alice]: https://en.wikipedia.org/wiki/Alice%27s_Adventures_in_Wonderland
% [carroll]: https://en.wikipedia.org/wiki/Lewis_Carroll
